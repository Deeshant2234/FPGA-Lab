\documentclass{article}
\usepackage[utf8]{inputenc}
\usepackage[T1]{fontenc}
\usepackage{tikz}

\title{\textbf{Assignment 1}  |\textbf{ FPGA Lab}}
\author{Deeshant Sharma [ EE21MTECH14002 ]}
\date{January 2022}

\begin{document}
\maketitle

\section{Question}

Using the truth table, state whether the following proposition is a tautology, contingency or a contradiction:
\begin{center}
    $ \neg( A \land B ) \lor ( \neg A => B ) $
\end{center}

\section{Solution}

\subsection{Tautology}
Truth Values are true for any combination of truth value of variables.

\subsection{Contradiction}
Truth Values are false for any combination of truth value of variables.

\subsection{Contingency}
Some Truth Values are true for some combination of truth value of variables and some truth value are false for  truth value combination of other variables.


\subsection{Truth Table}
\begin{displaymath}
\begin{array}{|c c c c c c|c|}
% |c c|c| means that there are three columns in the table and
% a vertical bar ’|’ will be printed on the left and right borders,
% and between the second and the third columns.
% The letter ’c’ means the value will be centered within the column,
% letter ’l’, left-aligned, and ’r’, right-aligned.
A & B & A \land B & \neg(A \land B) & \neg A & (\neg A => B ) & \neg (A \land B)(\neg A => B) \\ % Use & to separate the columns
\hline % Put a horizontal line between the table header and the rest.
\textbf{T} & \textbf{T} & T & F & F & T & \textbf{T}\\
\textbf{T} & \textbf{F} & F & T & F & T & \textbf{T}\\
\textbf{F} & \textbf{T} & F & T & T & T & \textbf{T}\\
\textbf{F} & \textbf{F} & F & T & T & F & \textbf{T}\\
\end{array}
\end{displaymath}

"Since for all combination of A and B given proposition gives output as \textbf{true} hence, given proposition is a \textbf{Tautology}" 


\end{document}