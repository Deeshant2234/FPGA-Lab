\documentclass{article}
\usepackage[utf8]{inputenc}
\usepackage[T1]{fontenc}
\usepackage{tikz}

\title{\textbf{Assignment 1}  |\textbf{ FPGA Lab}}
\author{Deeshant Sharma [ EE21MTECH14002 ]}
\date{January 2022}

\begin{document}
\maketitle

\section{Question}

Using the truth table, state whether the following proposition is a tautology, contingency or a contradiction:
\begin{center}
    $ ( A . B )' + (  A' => B ) $
\end{center}

\section{Solution}

\subsection{Tautology}
Truth Values are true(1) for any combination of truth value of variables.

\subsection{Contradiction}
Truth Values are false(0) for any combination of truth value of variables.

\subsection{Contingency}
Some Truth Values are true(1) for some combination of truth value of variables and some truth value are false(0) for  truth value combination of other variables.


\subsection{Truth Table}
\begin{displaymath}
\begin{array}{|c c c c c c|c|}
% |c c|c| means that there are three columns in the table and
% a vertical bar ’|’ will be printed on the left and right borders,
% and between the second and the third columns.
% The letter ’c’ means the value will be centered within the column,
% letter ’l’, left-aligned, and ’r’, right-aligned.
A & B & A . B & (A . B)' &  A' & (A' => B ) &  (A . B)'( A' => B) \\ % Use & to separate the columns
\hline % Put a horizontal line between the table header and the rest.
\textbf{1} & \textbf{1} & 1 & 0 & 0 & 1 & \textbf{1}\\
\textbf{1} & \textbf{0} & 0 & 1 & 0 & 1 & \textbf{1}\\
\textbf{0} & \textbf{1} & 0 & 1 & 1 & 1 & \textbf{1}\\
\textbf{0} & \textbf{0} & 0 & 1 & 1 & 0 & \textbf{1}\\
\end{array}
\end{displaymath}

"Since for all combination of A and B given proposition gives output as \textbf{true(1)} hence, given proposition is a \textbf{Tautology}" 


\end{document}