\documentclass{article}
\usepackage[utf8]{inputenc}
\usepackage[T1]{fontenc}
\usepackage{tikz}

\title{\textbf{Assignment 3}  |\textbf{ FPGA Lab}}
\author{Deeshant Sharma [ EE21MTECH14002 ]}
\date{April 2022}

\begin{document}
\maketitle

\section{Question}

We have to perform the problem presented in Assignment-1 on arduino and verify the output using assembly language.
\textbf{Also, Draw the truth table for the inputs and outputs for the given  expression}


\begin{center}
    $ ( A . B )' + (  A' => B ) $
\end{center}

\section{Operators Description}
\subsection{Binary Operator: AND}
The AND operator (symbolically: "\textbf{.}") also known as logical conjunction requires both A and B to be True(1) for the result to be True(1). All other cases result in False(0).

\subsection{Binary Operator: OR}
The OR operator (symbolically: "\textbf{+}") requires only one premise to be True(1) for the result to be True(1)

\subsection{Binary Operator: NOT}
The NOT operator is commonly represented by a [\textbf{'}]. It negates, or switches truth value.

\subsection{Conditional Operator: if-then}
Logical implication (symbolically:\textbf{ A → B} or "\textbf{=>}"), also known as “if-then”, results True(1) in all cases except the case T → F.This is logically equivalent to \textbf{A'+B}

\section{Solution}

\subsection{Tautology}
Truth Values are True(1) for any combination of truth value of variables.

\subsection{Contradiction}
Truth Values are False(0) for any combination of truth value of variables.

\subsection{Contingency}
Some Truth Values are True(1) for some combination of truth value of variables and some truth value are False(0) for  truth value combination of other variables.


\subsection{Truth Table}
\begin{displaymath}
\begin{array}{|c c c c c c|c|}
% |c c|c| means that there are three columns in the table and
% a vertical bar ’|’ will be printed on the left and right borders,
% and between the second and the third columns.
% The letter ’c’ means the value will be centered within the column,
% letter ’l’, left-aligned, and ’r’, right-aligned.
A & B & A . B & (A . B)' &  A' & (A' => B ) &  (A . B)'+( A' => B) \\ % Use & to separate the columns
\hline % Put a horizontal line between the table header and the rest.
\textbf{1} & \textbf{1} & 1 & 0 & 0 & 1 & \textbf{1}\\
\textbf{1} & \textbf{0} & 0 & 1 & 0 & 1 & \textbf{1}\\
\textbf{0} & \textbf{1} & 0 & 1 & 1 & 1 & \textbf{1}\\
\textbf{0} & \textbf{0} & 0 & 1 & 1 & 0 & \textbf{1}\\
\end{array}
\end{displaymath}

\section{Assembly Code}
\begin{verbatim}
.include "/home/ramesh720/m328Pdef.inc"

Start:
	ldi r17, 0b11000011 ; identifying input pins 10,11,12,13
	out DDRB,r17		; declaring pins as input
	ldi r17, 0b11111111 ;
	out PORTB,r17		; activating internal pullup for pins 10,11,12,13  
	in r17,PINB
	
	ldi r20,0b00000010
	rcall loopr
	
	ldi r21,0b00000001
	and r21,r17 ;w
	lsr r17
	ldi r22,0b00000001
	and r22,r17 ;z
	lsr r17
	ldi r23,0b00000001
	and r23,r17 ;b
	lsr r17
	ldi r24,0b00000001
	and r24,r17 ;a

	ldi r25,0b00000001
	
	mov r14,r23
	and r23,r24  ;r23 = a.b
	com r23      ;r23 = (a.b)'
	or r24,r14   ;r24 = a+b
	or r24,r23   ;r24 = (a+b) + (a.b)'



	ldi r20,0b00000010
	rcall loopl

	ldi r16, 0b00111100    ;identifying output pins 2,3,4,5
	out DDRD,r16		;declaring pins as output
	out PORTD,r24		;writing output to pins 2,3,4,5
	

	rjmp Start

loopr:	lsr r17
	dec r20
	brne loopr
	ret

loopl:	lsl r24
	dec r20
	brne loopl
	ret
\end{verbatim}

\section{Result}
Since for all combination of A and B given proposition gives output as \textbf{True(1)} hence, given proposition is a \textbf{Tautology}. 
\begin{verbatim}
The assignment has been completed and truth table isverified.
\end{verbatim}
Implemented the above truth table in Arduino. The inputs A,B equivalent are displayed on seven segment display and its corresponding output is displayed with LED.\\
Steps: \\
1. Login into ubuntu and go to avra-1.3.0 folder\\
2. In avra-1.3.0 folder open src folder and write program in assign3.asm\\
3. One program was written to display LHS truth table and other to display RHS truth table.\\
4. Compile the program to generate the hex file.\\
5. After generating the hex file save it on laptop and load it in Arduino usng XLoader.\\



\end{document}