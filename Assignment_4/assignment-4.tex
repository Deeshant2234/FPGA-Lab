\documentclass{article}
\usepackage[utf8]{inputenc}
\usepackage[T1]{fontenc}
\usepackage{tikz}

\title{\textbf{Assignment 4}  |\textbf{ FPGA Lab}}
\author{Deeshant Sharma [ EE21MTECH14002 ]}
\date{April 2022}

\begin{document}
\maketitle

\section{Question}

We have to perform the problem presented in Assignment-1 on arduino and verify the output using C language.
\textbf{Also, Draw the truth table for the inputs and outputs for the given  expression}


\begin{center}
    $ ( A . B )' + (  A' => B ) $
\end{center}

\section{Operators Description}
\subsection{Binary Operator: AND}
The AND operator (symbolically: "\textbf{.}") also known as logical conjunction requires both A and B to be True(1) for the result to be True(1). All other cases result in False(0).

\subsection{Binary Operator: OR}
The OR operator (symbolically: "\textbf{+}") requires only one premise to be True(1) for the result to be True(1)

\subsection{Binary Operator: NOT}
The NOT operator is commonly represented by a [\textbf{'}]. It negates, or switches truth value.

\subsection{Conditional Operator: if-then}
Logical implication (symbolically:\textbf{ A → B} or "\textbf{=>}"), also known as “if-then”, results True(1) in all cases except the case T → F.This is logically equivalent to \textbf{A'+B}

\section{Solution}

\subsection{Tautology}
Truth Values are True(1) for any combination of truth value of variables.

\subsection{Contradiction}
Truth Values are False(0) for any combination of truth value of variables.

\subsection{Contingency}
Some Truth Values are True(1) for some combination of truth value of variables and some truth value are False(0) for  truth value combination of other variables.


\subsection{Truth Table}
\begin{displaymath}
\begin{array}{|c c c c c c|c|}
% |c c|c| means that there are three columns in the table and
% a vertical bar ’|’ will be printed on the left and right borders,
% and between the second and the third columns.
% The letter ’c’ means the value will be centered within the column,
% letter ’l’, left-aligned, and ’r’, right-aligned.
A & B & A . B & (A . B)' &  A' & (A' => B ) &  (A . B)'+( A' => B) \\ % Use & to separate the columns
\hline % Put a horizontal line between the table header and the rest.
\textbf{1} & \textbf{1} & 1 & 0 & 0 & 1 & \textbf{1}\\
\textbf{1} & \textbf{0} & 0 & 1 & 0 & 1 & \textbf{1}\\
\textbf{0} & \textbf{1} & 0 & 1 & 1 & 1 & \textbf{1}\\
\textbf{0} & \textbf{0} & 0 & 1 & 1 & 0 & \textbf{1}\\
\end{array}
\end{displaymath}

\section{C Code}
\begin{verbatim}
#include <avr/io.h>
#include <util/delay.h>

 
int main (void)
{
 DDRD   |= 0b00000000;
 DDRB   |= ((1 << DDB5));
 int i,p,q,r,w,a,b,output;
  while (1) {

     i = PIND;
   
     r= i & 0b00000100;
     q= i & 0b00001000;
     p= i & 0b00010000;

	output=(a||b)||(!(a&&b));

if(output==1)
PORTB = ((1 <<  PB5));
else
PORTB = ((0 <<  PB5));

  }
  return 0;

}
\end{verbatim}

\section{Result}
Since for all combination of A and B given proposition gives output as \textbf{True(1)} hence, given proposition is a \textbf{Tautology}. 
\begin{verbatim}
The assignment has been completed and truth table is verified.
\end{verbatim}
Implemented the above truth table in Arduino. Output for different input combinations of A,B are displayed with Arduino builtin LED.\\

\end{document}
